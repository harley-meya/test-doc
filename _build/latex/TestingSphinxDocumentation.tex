%% Generated by Sphinx.
\def\sphinxdocclass{report}
\documentclass[letterpaper,10pt,english]{sphinxmanual}
\ifdefined\pdfpxdimen
   \let\sphinxpxdimen\pdfpxdimen\else\newdimen\sphinxpxdimen
\fi \sphinxpxdimen=.75bp\relax

\usepackage[utf8]{inputenc}
\ifdefined\DeclareUnicodeCharacter
 \ifdefined\DeclareUnicodeCharacterAsOptional
  \DeclareUnicodeCharacter{"00A0}{\nobreakspace}
  \DeclareUnicodeCharacter{"2500}{\sphinxunichar{2500}}
  \DeclareUnicodeCharacter{"2502}{\sphinxunichar{2502}}
  \DeclareUnicodeCharacter{"2514}{\sphinxunichar{2514}}
  \DeclareUnicodeCharacter{"251C}{\sphinxunichar{251C}}
  \DeclareUnicodeCharacter{"2572}{\textbackslash}
 \else
  \DeclareUnicodeCharacter{00A0}{\nobreakspace}
  \DeclareUnicodeCharacter{2500}{\sphinxunichar{2500}}
  \DeclareUnicodeCharacter{2502}{\sphinxunichar{2502}}
  \DeclareUnicodeCharacter{2514}{\sphinxunichar{2514}}
  \DeclareUnicodeCharacter{251C}{\sphinxunichar{251C}}
  \DeclareUnicodeCharacter{2572}{\textbackslash}
 \fi
\fi
\usepackage{cmap}
\usepackage[T1]{fontenc}
\usepackage{amsmath,amssymb,amstext}
\usepackage{babel}
\usepackage{times}
\usepackage[Bjarne]{fncychap}
\usepackage[dontkeepoldnames]{sphinx}

\usepackage{geometry}

% Include hyperref last.
\usepackage{hyperref}
% Fix anchor placement for figures with captions.
\usepackage{hypcap}% it must be loaded after hyperref.
% Set up styles of URL: it should be placed after hyperref.
\urlstyle{same}
\addto\captionsenglish{\renewcommand{\contentsname}{Contents:}}

\addto\captionsenglish{\renewcommand{\figurename}{Fig.}}
\addto\captionsenglish{\renewcommand{\tablename}{Table}}
\addto\captionsenglish{\renewcommand{\literalblockname}{Listing}}

\addto\captionsenglish{\renewcommand{\literalblockcontinuedname}{continued from previous page}}
\addto\captionsenglish{\renewcommand{\literalblockcontinuesname}{continues on next page}}

\addto\extrasenglish{\def\pageautorefname{page}}

\setcounter{tocdepth}{1}



\title{Testing Sphinx Documentation Documentation}
\date{Nov 10, 2017}
\release{2.0.0}
\author{Meya.ai}
\newcommand{\sphinxlogo}{\vbox{}}
\renewcommand{\releasename}{Release}
\makeindex

\begin{document}

\maketitle
\sphinxtableofcontents
\phantomsection\label{\detokenize{index::doc}}



\chapter{Test2}
\label{\detokenize{page2:test2}}\label{\detokenize{page2::doc}}\label{\detokenize{page2:welcome-to-testing-sphinx-documentation-s-documentation}}
\sphinxstyleemphasis{This is in italics}

\sphinxstylestrong{This is bold}

\sphinxcode{This is code}
\begin{itemize}
\item {} 
This is a bulleted list.

\item {} 
With two items, the second
item uses two lines.

\end{itemize}
\begin{enumerate}
\item {} 
This is a numbered list.

\item {} 
It has two items too.

\item {} 
This is a numebred list.

\item {} 
It has two items too.

\end{enumerate}
\begin{itemize}
\item {} 
list item 1

\item {} 
list item 2
\begin{itemize}
\item {} 
nest list item 1

\item {} 
nest list item 2

\end{itemize}

\item {} 
list item 3

\end{itemize}
\begin{description}
\item[{term (up to a line of text)}] \leavevmode
Definition of the term, which is indented.

and can be multipe paragraphs.

\item[{next term}] \leavevmode
Description.

\end{description}

These lines are
broken exactly like in
the source file.

\begin{DUlineblock}{0em}
\item[] These lines are
\item[] broken exactly like in
\item[] the source file.
\end{DUlineblock}

This is a normal paragraph. The next paragraph is code

\begin{sphinxVerbatim}[commandchars=\\\{\}]
\PYG{n}{It} \PYG{o+ow}{is} \PYG{o+ow}{not} \PYG{n}{processed} \PYG{o+ow}{in} \PYG{n+nb}{any} \PYG{n}{way}\PYG{p}{,} \PYG{k}{except}
\PYG{n}{that} \PYG{n}{the} \PYG{n}{idnentaion} \PYG{o+ow}{is} \PYG{n}{removed}\PYG{o}{.}

\PYG{n}{It} \PYG{n}{can} \PYG{n}{span} \PYG{n}{multiple} \PYG{n}{lines}\PYG{o}{.}
\end{sphinxVerbatim}

This is normal text again.

Here is a grid table:


\begin{savenotes}\sphinxattablestart
\centering
\begin{tabulary}{\linewidth}[t]{|T|T|T|T|}
\hline
\sphinxstylethead{\sphinxstyletheadfamily 
Header row, col 1
\unskip}\relax &\sphinxstylethead{\sphinxstyletheadfamily 
Header 2
\unskip}\relax &\sphinxstylethead{\sphinxstyletheadfamily 
Header 3
\unskip}\relax &\sphinxstylethead{\sphinxstyletheadfamily 
Header 4
\unskip}\relax \\
\hline
body row 1, col 1
&
column 2
&
column 3
&
column 4
\\
\hline
\end{tabulary}
\par
\sphinxattableend\end{savenotes}

not a table.


\begin{savenotes}\sphinxattablestart
\centering
\begin{tabulary}{\linewidth}[t]{|T|T|T|}
\hline
\sphinxstylethead{\sphinxstyletheadfamily 
A
\unskip}\relax &\sphinxstylethead{\sphinxstyletheadfamily 
B
\unskip}\relax &\sphinxstylethead{\sphinxstyletheadfamily 
A and B
\unskip}\relax \\
\hline
False
&
False
&
False
\\
\hline
True
&
False
&
False
\\
\hline
False
&
True
&
False
\\
\hline
True
&
True
&
True
\\
\hline
\end{tabulary}
\par
\sphinxattableend\end{savenotes}

not a table.

\sphinxurl{http://www.google.com}

Go to \sphinxhref{http://www.google.com}{Google} to search stuff.


\chapter{1   Welcome to page 3!}
\label{\detokenize{page3::doc}}\label{\detokenize{page3:welcome-to-page-3}}\begin{quote}

No matter where you go, there you are.

\begin{flushright}
---Buckaroo Banzai
\end{flushright}
\end{quote}


\section{1.1   Section 1}
\label{\detokenize{page3:section-1}}
\begin{sphinxVerbatim}[commandchars=\\\{\}]
\PYG{k}{def} \PYG{n+nf}{my\PYGZus{}function}\PYG{p}{(}\PYG{p}{)}\PYG{p}{:}
    \PYG{l+s+s2}{\PYGZdq{}}\PYG{l+s+s2}{just a test}\PYG{l+s+s2}{\PYGZdq{}}
    \PYG{n+nb}{print} \PYG{l+m+mi}{8}\PYG{o}{/}\PYG{l+m+mi}{2}
\end{sphinxVerbatim}

\begin{sphinxadmonition}{danger}{Danger:}
Beware killer rabbits!
\end{sphinxadmonition}

\begin{sphinxVerbatim}[commandchars=\\\{\}]
\PYG{g+gp}{\PYGZgt{}\PYGZgt{}\PYGZgt{} }\PYG{l+m+mi}{1} \PYG{o}{+} \PYG{l+m+mi}{1}
\PYG{g+go}{2}
\end{sphinxVerbatim}


\section{1.2   Section 2}
\label{\detokenize{page3:section-2}}
\begin{sphinxadmonition}{attention}{Attention:}
Save your file!
\end{sphinxadmonition}

\begin{sphinxShadowBox}
\sphinxstylesidebartitle{Sidebar Title}
\sphinxstylesidebarsubtitle{Optional subtitle}

Blah.
\end{sphinxShadowBox}

\begin{sphinxVerbatim}[commandchars=\\\{\},numbers=left,firstnumber=10,stepnumber=1]
\PYG{k}{def} \PYG{n+nf}{my\PYGZus{}func}\PYG{p}{(}\PYG{p}{)}\PYG{p}{:}
      \PYG{k}{print} \PYG{l+s+s2}{\PYGZdq{}}\PYG{l+s+s2}{Hello, world!}\PYG{l+s+s2}{\PYGZdq{}}

      \PYG{k}{print} \PYG{l+m+mi}{1}\PYG{o}{+} \PYG{l+m+mi}{1}
      \PYG{k}{return} \PYG{n+nb+bp}{True}
\end{sphinxVerbatim}


\subsection{1.2.1   Section A}
\label{\detokenize{page3:section-a}}\begin{quote}

Text.
\end{quote}


\subsubsection{1.2.1.1   Section a}
\label{\detokenize{page3:id1}}
\begin{sphinxadmonition}{caution}{Caution:}
I am moody.
\end{sphinxadmonition}


\subsection{1.2.2   Section B}
\label{\detokenize{page3:section-b}}
\begin{sphinxadmonition}{error}{Error:}
Go back!
\end{sphinxadmonition}

\begin{sphinxShadowBox}
\sphinxstyletopictitle{Topic Title}

Yay!
\end{sphinxShadowBox}

\begin{sphinxadmonition}{hint}{Hint:}
Save first!
\end{sphinxadmonition}

\begin{sphinxadmonition}{important}{Important:}
Do this!
\end{sphinxadmonition}

\begin{sphinxadmonition}{note}{Note:}
Buy milk
\end{sphinxadmonition}

\begin{sphinxadmonition}{tip}{Tip:}
Mix with water.
\end{sphinxadmonition}

\begin{sphinxadmonition}{warning}{Warning:}
Cross road.
\end{sphinxadmonition}


\chapter{Indices and tables}
\label{\detokenize{index:indices-and-tables}}\begin{itemize}
\item {} 
\DUrole{xref,std,std-ref}{genindex}

\item {} 
\DUrole{xref,std,std-ref}{modindex}

\item {} 
\DUrole{xref,std,std-ref}{search}

\end{itemize}



\renewcommand{\indexname}{Index}
\printindex
\end{document}